\documentclass[journal=jacsat,manuscript=article,layout=twocolumn,12pt]{achemso}
\usepackage[version=3]{mhchem} % Formula subscripts using \ce{}
\usepackage[T1]{fontenc}       % Use modern font encodings
\usepackage{amsmath}
% NB added command for in line cite
\newcommand{\onlinecite}[1]{\hspace{-1 ex} \nocite{#1}\citenum{#1}} 
% 2 column equations
\usepackage{widetext, widetable}
%
\author{Z.~Levey}
\author{B.~A.~Laws}
\email{B.Laws@unsw.edu.au}
\author{K.~Nauta} 
\author{T.~Schmidt}

\affiliation{School of Chemistry, University of New South Wales, Sydney NSW 2052, Australia}
\title{Formation of PhenRad via CH$_2$ ring insertion...}
\abbreviations{PES,PAD,EA,eKE,FWHM,VMI}
\begin{document} 
\begin{abstract} 
This is the manuscripts abstract...
\end{abstract} 
%\begin{tocentry}
%\includegraphics[width=1\textwidth]{Figures/TOC}
%\end{tocentry}
\section{Introduction}
Intro...~\cite{por20,law17}
\section{Results}
\section{Discussion}
\section{Conclusion}
\section{Experimental Details}
\begin{acknowledgement}
	This research was supported by the Australian Research Council Discovery
	Project Grant DP160102585. The author's thank Andrei Sanov for discussion on his modified Cooper-Zare anisotropy model.
\end{acknowledgement}


% Create the reference section using BibTeX: 
\bibliography{PhenRad}

%\newpage
%\onecolumn
%\subsection{TOC Graphic}
%\vspace{2ex}
%\begin{center}
%	\includegraphics[width=8.5cm]{Figures/TOC}
%\end{center}


\end{document}
